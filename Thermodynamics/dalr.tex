\documentclass[a4paper,12pt]{article}
\usepackage{amsmath, amssymb, graphicx}
\graphicspath{ {D:/Education/ATMOCN_630/hw3/} }

\DeclareMathAlphabet{\mathcal}{OMS}{cmsy}{m}{n}
\SetMathAlphabet{\mathcal}{bold}{OMS}{cmsy}{b}{n}
\newcommand{\bigO}{\mathcal{O}}


\begin{document}

\title{\vspace{-4.0cm}Dry Adiabatic Lapse Rate}
\author{Ian Beckley
\\University of Wisconsin-Madison}

\date{28 Oct. 2020}

\maketitle
A \emph{lapse rate}, $\Gamma$, is defined as the rate at which temperature \emph{decreases} with height. 

\begin{align}
\Gamma = -\frac{dT}{dz} \sim \text{ K km}^{-1}
\end{align}

In order to derive the \emph{dry adiabatic lapse rate} we must first recall an application of the 2nd law of thermodynamics. That is,

\begin{align}
\rho c_p \frac{dT}{dt} = \frac{dp}{dt} + H
\end{align}

Recall that H represents diabatic processes diabatic processes, specifically, $H = Q - (\nabla \cdot \vec F) + \epsilon$, that is, heating related to chemical reactions and phase changes (latent heating), radiative heating and the conversion of kinetic energy to internal energy (thermalization).

By definition, $H=0$ in adiabatic processes. This allows a substantial simplification of equation (2).

\begin{align}
\rho c_p \frac{dT}{dt} = \frac{dp}{dt}\\
dt \rho c_p \frac{dT}{dt} = \frac{dp}{dt} dt\\
\rho c_p dT = dp\\
\frac{dT}{dp} = \frac{1}{\rho c_p}
\end{align}

Equation (6) defines the derivative of temperature with resepct to temperature, however we would like to derive temperature with resepect to height. Recall hydrostatic balance,

\begin{align}
\frac{dp}{dz} = -\rho g\\
dp = -\rho g \, dz
\end{align}

Substituting (8) into the denominator of the LHS in (6),

\begin{align*}
\frac{dT}{-\rho g \, dz} = \frac{1}{\rho c_p}\\
\frac{dT}{dz} = \frac{-\rho g}{\rho c_p}\\
\frac{dT}{dz} = -\frac{g}{c_p}
\end{align*}

Thus, the dry adiabatic lapse rate is,

\begin{align*}
\boxed{\Gamma_d \approx 9.8 \text{ K km}^{-1}}
\end{align*}

\end{document}