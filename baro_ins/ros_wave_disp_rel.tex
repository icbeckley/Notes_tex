\documentclass[a4paper,12pt]{article}
\usepackage{amsmath, amssymb, graphicx}
\graphicspath{ {D:/Education/ATMOCN_630/hw3/} }

\DeclareMathAlphabet{\mathcal}{OMS}{cmsy}{m}{n}
\SetMathAlphabet{\mathcal}{bold}{OMS}{cmsy}{b}{n}
\newcommand{\bigO}{\mathcal{O}}


\begin{document}

\title{\vspace{-4.0cm} Barotropic Rossby Wave Dispersion Relation}
\author{Ian Beckley
\\University of Wisconsin-Madison}

\date{24 Nov. 2020}

\maketitle

\subsection*{Assumptions}
\

\begin{itemize}
	\item Conservation of absolute vorticity, that is, $\frac{d}{dt}(\zeta + f) = 0$
	\item Ignore stretching term, $\frac{dw}{dz} = 0$
	\item Uniform zonal flow ($\bar{\zeta} = 0$, $\bar{v} = 0$, $\bar{w} = 0$)
	\item Linearization (ignore $u\prime\frac{\partial \zeta\prime}{\partial x}$, $v\prime\frac{\partial \zeta\prime}{\partial y}$)
\end{itemize}

\subsection*{Linearization of the conservation of absolute vorticity}
We begin with the conservation of absolute vorticity and expand the total derivative into its Eulerian and advective components. 

\begin{align}
\frac{d}{dt}(\zeta + f) = 0\\
(\frac{\partial}{\partial t} + u\frac{\partial}{\partial x} + v\frac{\partial}{\partial y})(\zeta + f) = 0
\end{align}

Recall that under uniform zonal flow we may drop $\bar{\zeta} = 0$ and $\bar{v} = 0$ when linearizing.

\begin{align}
(\frac{\partial}{\partial t} + \bar{u}\frac{\partial}{\partial x} + u\prime\frac{\partial}{\partial x} + v\prime\frac{\partial}{\partial y})(\zeta \prime + f) = 0
\end{align}

Ignore $u\prime\frac{\partial \zeta\prime}{\partial x}$ and $v\prime\frac{\partial \zeta\prime}{\partial y}$ and recall that f only varies with latitude ($\beta$) whilst distributing the linearized derivative operand into the perturbation absolute vorticity.

\begin{align}
(\frac{\partial}{\partial t} + \bar{u}\frac{\partial}{\partial x})\zeta\prime + v\prime \beta = 0
\end{align}

\subsection*{Spatial and temporal derivatives of the perturbation streamfunction}

We now pause our evaluation of the consevation of absolute vorticity, and recall that $\psi\prime = \psi_0 e^{i(kx-\omega t)}$, which assumes a sinusoidal wave pattern due to the streamfunction perturbation.

Taking the first derivative of the streamfunction perturbation, $\psi\prime$

\begin{align}
\frac{\partial \psi\prime}{\partial x} = \frac{\partial(\psi_0 e^{i(kx-\omega t)})}{\partial x}\\
= \psi_0 e^{i(kx-\omega t)}) \cdot ik\\
= \psi\prime \, ik
\end{align}

Additionally, recall that the zonal derivative of the wavefunction perturbation is equal to the meridional flow perturbation, or

\begin{align}
\psi \prime_x = v\prime\\
v\prime = \psi\prime \, ik
\end{align}

Taking the second derivative of the perturbation streamfunction with respect to x,

\begin{align}
\psi\prime_{xx} = \psi\prime ik \cdot ik\\
= -k^2\psi\prime
\end{align}

Recall that $\zeta\prime = \frac{\partial v\prime}{\partial x} - \frac{\partial u}{\partial y}$, or,

\begin{align}
\zeta\prime = \psi\prime_{xx} - \psi\prime_{yy}
\end{align}

However, recall that $\psi\prime = \psi_0 e^{i(kx - wt)}$ and it is obvious that $\psi\prime$ does not vary in $y$ ($\psi\prime_{yy} = 0$). Thus,

\begin{align}
\zeta\prime = \psi\prime_{xx}
\end{align}

Finally, we take the derivative of $\psi\prime$ with respect to time,

\begin{align}
\frac{\partial \psi\prime}{\partial t} = \frac{\partial(\psi_0 e^{i(kx-\omega t)})}{\partial t}\\
= \psi\prime \cdot -i\omega
\end{align}

Thus, the linearized total derivative operand, when applied to $\psi\prime$ is,

\begin{align}
(\frac{\partial}{\partial t} + \bar{u}\frac{\partial}{\partial x})\psi\prime = (\frac{\partial \psi\prime}{\partial t} + \bar{u}\frac{\partial \psi\prime}{\partial x})\\
= (-i\omega \psi\prime + ik\psi\prime)\\
= (-i\omega + ik)\psi\prime
\end{align}

\subsection*{Return to conservation of absolute vorticity}

We now have substitutions (see (11/13), (9), and (18) for many of the terms seen in the linearized conservation of absolute vorticity (4). As a reminder, we begin with (4).

\begin{align*}
(\frac{\partial}{\partial t} + \bar{u}\frac{\partial}{\partial x})\zeta\prime + v\prime \beta = 0
\end{align*}
\begin{align}
-(-i\omega + i\bar{u}k)k^2\psi\prime + ik\beta\psi\prime = 0
\end{align}

\subsection*{Frequency, Phase Speed and Group Velocity}

Isolating $\omega$

\begin{align}
-(-i\omega + i\bar{u}k)k^2\psi\prime + ik\beta\psi\prime = 0\\
(-i\omega + i\bar{u}k)k^2\psi\prime =  ik\beta\psi\prime\\
(-i\omega + i\bar{u}k)k^2 =  ik\beta\\
(-i\omega + i\bar{u}k)k =  i\beta\\
-i\omega k + i\bar{u}k^{2} = i\beta\\
-\omega k + \bar{u}k^{2} = \beta\\
-\omega k = \beta - \bar{u}k^{2}\\
-\omega = \frac{\beta}{k} - \bar{u}k\\
\boxed{\omega = \bar{u}k - \frac{\beta}{k}}
\end{align}

Divide (28) by $k$ in order to derive the celerity of an individual Rossby wave, $c_x$

\begin{align}
\frac{\omega}{k} = \frac{\bar{u}k}{k} - \frac{\beta}{k^2}\\
\boxed{c_x = \bar{u} - \frac{\beta}{k^2}}
\end{align}

Recall that the group velocity, $G_x$ is the derivative of $\omega$ with respect to $k$, thus,

\begin{align}
\frac{\partial \omega}{\partial k} = \bar{u} + \frac{\beta}{k^2}\\
\boxed{G_x = \bar{u} + \frac{\beta}{k^2}}
\end{align}


\subsection*{Implications}

\begin{itemize}
	\item Rossby waves are dispersive in $k$ ($\omega$ explicitely depends on $k$)
	\item $-\frac{\beta}{k^2}$ is the Rossby wave propagation speed, thus, individual transverse waves travel upstream
	\item $\frac{\beta}{k^2}$ is the Rossby \emph{packet energy} propagation speed, thus, energy propagates downstream
	\item High wave-number waves (large $k$) travel westward quickly ($-\frac{\beta}{k^2} << \bar{u}$) (e.g. short-wave troughs can round the base of long-wave troughs)
	\item Low wave-number waves (small $k$) can be quasi-stationary or even travel upstream ($-\frac{\beta}{k^2} \approx \bar{u}$) (e.g. 500 hPa 'blocks' can retrograde within the flow)
	\item The above is an alternative view of the quasi-geostrophic equation for heigh falls ($\chi$ equation). For example, the forcing for height falls associated with a cut-off low (or blocking) high is usually dominated by planetary vorticity advection within the meridional flow associated with such a feature.
\end{itemize}
	

\end{document}
