\documentclass[a4paper,12pt]{article}
\usepackage{amsmath, amssymb, graphicx}
\graphicspath{ {D:/Education/ATMOCN_630/hw3/} }

\DeclareMathAlphabet{\mathcal}{OMS}{cmsy}{m}{n}
\SetMathAlphabet{\mathcal}{bold}{OMS}{cmsy}{b}{n}
\newcommand{\bigO}{\mathcal{O}}


\begin{document}

\title{\vspace{-4.0cm}Geostrophic Approximation}
\author{Ian Beckley
\\University of Wisconsin-Madison}

\date{12 Jan. 2021}

\maketitle

\subsection*{Horizontal Momentum Equations}
Recall that $-\delta p = g\rho\delta z$ (hydrostatic balance). The horizontal equations of motion (see "Navier-Stokes on a Rotating Sphere") can be rewritten in terms of geopotential ($\Phi = gz$).

\begin{align}
\frac{\vec{V_h}}{dt} = -f\vec{k} \times \vec{V} - \nabla_p \Phi + \vec{F}
\end{align}

The expanded coriolis term is

\begin{align}
-2\Omega \times \vec{V} = -2\pi\begin{vmatrix} \hat{i} & \hat{j} & \hat{k} \\ 0 & 0 & -f \\  u & v & w \end{vmatrix}\\
-fv \hat{i} + fu \hat{j}
\end{align}

such that horizontal equations of motion can be written

\begin{align}
\frac{du}{dt} = -fv - \frac{\partial \Phi}{\partial x} + F_x\\
\frac{dv}{dt} = fu + \frac{\partial \Phi}{\partial y} + F_y
\end{align}

\subsection*{Geostrophic Wind}
Forgoing the friction term is acceptable aloft of the boundary layer, and we can again form geostrophic balance and define the geostrophic winds.

\begin{align}
\frac{du}{dt} = 0 = -fv - \frac{\partial \Phi}{\partial x}\\
fv = -\frac{d\phi}{dx}\\
\boxed{v_g = -\frac{1}{f}\frac{\partial \Phi}{\partial x}}
\end{align}

\begin{align}
\frac{dv}{dt} = 0 = fu - \frac{\partial\Phi}{\partial y}\\
fu = \frac{d\Phi}{dy}\\
\boxed{u_g = \frac{1}{f}\frac{\partial \Phi}{\partial y}}
\end{align}

Recall that these can also be evaluated with the pressure gradient via substitution of hydrostatic balance ($\delta p = -\rho g\delta z$). This may be easier for a surface pressure map, while aloft the height coordinate version will be more useful. 

\subsection*{Non-Divergence Theorem}
Another rendition of the non-divergence theorem is shown within the "Navier-Stokes on a Rotating Sphere" notes using the geostrophic streamfunction. The process here is similar.

\begin{align}
\nabla \cdot \vec{V_g} = \frac{\partial u_g}{\partial x} + \frac{\partial v_g}{\partial y}
\end{align}

Substituting (8) and (11),

\begin{align}
-\frac{1}{f}\frac{\partial}{\partial x}\frac{\partial \Phi}{\partial y} + \frac{1}{f}\frac{\partial}{\partial y}\frac{\partial \Phi}{\partial x} = 0
\end{align}

Thus, the geostrophic wind is non-divergent. Without convergence and divergence, however, there could be no vertical motion field. It follows that some portion of the total wind must be ageostrophic and responsible for divergence and convergence on our planet.

\subsection*{Rossby Number}
We can use the Rossby number to evaluate the validity of the geostrophic approximation. The Rossby number is defined as the ratio of the horizontal acceleration and coriolis force and alternatively as the ratio between the flow speed and product of the coreolis parameter and length scales.

\begin{align}
\boxed{R_0 = \frac{|\frac{du}{dt}|}{fv} \sim \frac{U}{fL}}
\end{align}

On the synoptic scale $f \sim 10^-4$, $L \sim 10^6$ and $U \sim 10$ such that $R_0 << 1$. Under this criteria $\vec{V}$ is well approximated as $\vec{V_g}$. If $R \sim 1$ it should be clear that the pressure gradient force is dominating the coriolis force, and the geostrophic approximation should not be made.
\end{document}