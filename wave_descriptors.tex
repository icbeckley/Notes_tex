\documentclass[a4paper,12pt]{article}
\usepackage{amsmath, amssymb, graphicx}
\graphicspath{ {D:/Education/ATMOCN_630/hw3/} }

\DeclareMathAlphabet{\mathcal}{OMS}{cmsy}{m}{n}
\SetMathAlphabet{\mathcal}{bold}{OMS}{cmsy}{b}{n}
\newcommand{\bigO}{\mathcal{O}}


\begin{document}

\title{\vspace{-4.0cm}Waves}
\author{Ian Beckley
\\University of Wisconsin-Madison}

\date{2 Dec. 2020}

\maketitle

\section*{Descriptors}
\subsection*{What is a Wave?}
\begin{itemize}
	\item Quasi-periodic
	\item Transfer energy and momentum at large distances without transfer of fluid particles
\end{itemize}

\subsection*{Free/Forced}
\begin{itemize}
	\item Free waves are resonant normal modes, excited by weak random motions in the atmosphere
	\item e.g., Hirota and Hirooka (1984), '5-Day Wave'
	\item Forced waves require continuous forcing at compatible space ant time scales
	\item e.g., Tides
\end{itemize}

\subsection*{Internal/External}
\begin{itemize}
	\item Internal: wave amplitude maximizes in the interior
	\item Sinusoidal form for phase variation $\sim e^{i m_r z}$ which indicates sinusoidal variation in ther vertical
	\item e.g., synoptic Rossby waves, internal gravity waves\\
	\item External: wave amplitude maximizes at the edge ('evanescent waves' $\sim$ amplitude decays in space)
	\item wave energy decreases away from the boundary $\sim e^{-m_i z}$
	\item e.g.,  surface water waves (wave impact dampens away from the density gradient)
\end{itemize}

\subsection*{Stationary/Travelling}
\begin{itemize}
	\item Stationary $c_r = 0$
	\item e.g. 'standing wave' (fixed node seiche, guitar string, etc)
	\item Travelling $c_r \neq 0$
\end{itemize}

\subsection*{Steady/Transient}
\begin{itemize}
	\item Steady $c_i = 0$, fixed amplitude
	\item Transient $c_i \neq 0$, amplitude varies in time (growth or decay)
\end{itemize}

\subsection*{Linear/Non-linear}
\begin{itemize}
	\item Linear (infinitesimal, theoretical)
	\item Nonlinear (finite amplitude, observable)
\end{itemize}

Take the Eulerian and advective form of the Lagrangian time rate of change of $u$. The flow is 'self-steepening' in that over faster flow will 'catch up to' slower flow and, subsequently, break. This process is fundamental to chaos theory. Note, however, that if $\overline{u}$ is constant to first order, the total derivative is linearized ($u = \overline{u} + u'$).

\begin{align*}
\frac{\partial u}{\partial t} + u\frac{\partial u}{\partial x} = 0 \text{ nonlinear advection, self-steepening with a tendency to break}\\
\text{If} \frac{|u'|}{|\overline{u}|} \ll 1 \text{ where } u = \overline{u} + u' \text{ then}\\
\frac{\partial u'}{\partial t} + \overline{u}\frac{\partial u'}{\partial x} = 0 \text{ advection linear in eddy amplitude $u'$, linearized operator}
\end{align*}

\subsection*{Dispersive/Non-dispersive}
\begin{itemize}
	\item Dispersive: propagation speed depends on wavelength
	\item e.g., Rossby waves ($c = -\frac{\beta}{k^2} \propto L_x^2$)
	\item Non-dispersive: all waves travel at the same speed, regardless of wavelength
	\item e.g., linear shallow water waves, $c = \sqrt{gh}$; linear sound waves, $c = \sqrt{\gamma RT}$
\end{itemize}

\section*{Dispersion relation}
\begin{itemize}
	\item A \emph{dispersion relation} relates the time and space scales of a given wve type to physical parameters
	\item Expressed as the functional dependence of frequency $\omega$ on size and physics ($\omega = \omega(\vec{k}\text{, physics})$)
	\item If $\omega(k)$, waves travel at different speeds
	\item Can also show how energy and wave crests move relative to one another
	\item A \emph{critical surface} is defined where $c = \overline{u}$. This determines where wave energy emerges and cannot pass\
	\item \emph{phase speed} ($c = \frac{\omega}{k}$) indicates how fast a wave crest or trough moves
	\item \emph{group speed} ($G= \frac{\partial \omega}{\partial k}$) indicates how fast energy is moving
\end{itemize}

\section*{Wave Definitions}
\subsection*{Phase}
\emph{phase} is defined as $\theta = kx + ly+ mz - \omega t$, which is the definition of a plane in 3D place with a temporal dependence (wave on a given phase can travel relative to the three axis). Phase increases in the direction of propagation...or phase decreases in time as a wave crest passes (hence negative sign on $\omega t$ term.\\

\subsection*{Disturbance}
Take a generalized disturbance: $\psi = \text{Re}\left[Ce^{i\theta}\right] = \text{Re}\left[(C_r + iC_i)(\cos{\theta}+ i\sin{\theta})\right] = C_r \cos{\theta}- C_i\sin{\theta} = A\sin{\theta}+B\sin{\theta}$.

\subsection*{Wavenumber Vector}

\begin{align}
\vec{k} = \nabla \theta = \hat{i}\frac{\partial \theta}{\partial x} + \hat{j}\frac{\partial \theta}{\partial y} + \hat{k} \frac{\partial \theta}{\partial z}
\end{align}

Recall over one wavelength $\partial \theta = 2\pi$, thus, $k = \frac{2\pi}{k}$ (same for meridional and vertical wave numbers)

\begin{align*}
|\vec{k}| = \sqrt{k^2 + l^2 + m^2}
\end{align*}

Since $\vec{k} = \nabla \theta$, $k \perp \theta$ pointing towards increasing $\theta$.\\

$\therefore \,$ phase decreases with time at a point

\subsection*{Frequency}
$\omega = -\frac{\partial \theta}{\partial t}$ 'how fast $\theta$ varies in time. When $\omega > 0$ the wave is travelling to the east. When $\omega < 0$ the wave is travelling to the west. Though, we don't really talk about the sign of wave periodicity.

In $\tau$ (one period), $\partial \theta  = 2\pi$, thus, $|\omega| = \frac{2\pi}{\tau}$

\subsection*{Conservation of Wave Crests}
The local rate of change of the wavenumber vector and the frequency gradient sum to zero, or,

\begin{align*}
\frac{\partial \vec{k}}{\partial t} + \nabla \omega = 0\\
\frac{\partial}{\partial t}(\frac{\partial \theta}{\partial x}) + \frac{\partial}{\partial x}(-\frac{\partial \theta}{\partial t}) = 0
\end{align*}

This is easily interpreted through an example. If $\omega$ is decreasing to the east ($\nabla \omega > 0$) then $\vec{k}$ must decrease in time (longer waves reaching the point).

\subsection*{Phase Velocity}
This is the rate that a point on a phase surface travels in the direction of $\vec{k}$.

\begin{align*}
\vec{c_p} = \frac{\omega}{|\vec{k}|}\frac{\vec{k}}{|\vec{k}|} = \frac{\omega}{k^2 + l^2 + m^2}(k\hat{i} + l\hat{j} + m\hat{k})
\end{align*}

$\vec{c_p} \perp \theta$ and $\vec{c_p} \parallel \vec{k}$

In 1D ($l = 0$, $m=0$), $c_p = {\omega k}{k^2}$



\end{document}